%!TEX TS-program = xelatex 
%!TEX TS-options = -synctex=1 -output-driver="xdvipdfmx -q -E"
%!TEX encoding = UTF-8 Unicode
%
%  ganson
%
%  Created by Mark Eli Kalderon on 2015-06-09.
%  Copyright (c) 2015. All rights reserved.
%

\documentclass[12pt]{article} 

% Definitions
\newcommand\mykeywords{Aristotle, perception}
\newcommand\myauthor{Mark Eli Kalderon}

% Packages
\usepackage{geometry} \geometry{a4paper} 
\usepackage{url}
\usepackage{txfonts}
\usepackage{color}
\usepackage{enumerate}
\definecolor{gray}{rgb}{0.459,0.438,0.471}
\usepackage{setspace}
% \doublespace % Uncomment for doublespacing if necessary
\usepackage{epigraph} % optional

% XeTeX
\usepackage[cm-default]{fontspec}
\usepackage{xltxtra,xunicode}
\defaultfontfeatures{Scale=MatchLowercase,Mapping=tex-text}
\setmainfont{Hoefler Text}

% Bibliography
\usepackage[round]{natbib}

% Title Information
\title{Reply to Ganson}
\author{\myauthor} 
\date{} % Leave blank for no date, comment out for most recent date

% PDF Stuff
\usepackage[plainpages=false, pdfpagelabels, bookmarksnumbered, backref, pdftitle={Form Without Matter}, pagebackref, pdfauthor={\myauthor}, pdfkeywords={\mykeywords}, xetex, colorlinks=true, citecolor=gray, linkcolor=gray, urlcolor=gray]{hyperref} 

%%% BEGIN DOCUMENT
\begin{document}

% Title Page
\maketitle
% \begin{abstract} % optional
% \noindent
% \end{abstract} 
% \vskip 2em \hrule height 0.4pt \vskip 2em
\epigraph{Like trains of cars on tracks of plush \\
I hear the level bee: \\
A jar across the flowers goes, \\
Their velvet masonry \\

\hfill \break

Withstands until the sweet assault \\
Their chivalry consumes, \\
While he, victorious, tilts away \\
To vanquish other blooms.}{Emily Dickinson, \emph{The Bee}} % optional; make sure to uncomment \usepackage{epigraph}

% Layout Settings
\setlength{\parindent}{1em}

% Main Content

I am very grateful to Todd Ganson for his comments. Ganson presses two problems but spends the bulk of his time on the second. I will ignore his first problem concerning illusion and focus, after an initial set up, exclusively on the second problem concerning primitive agency. I do this not because I believe that the first problem is unimportant. Far from it. One of the disappointing aspects of Hilary Putnam's \citeyearpar{Putnam:2012eu} reply to Victor Caston \citeyearpar{Caston:1998nx} is that he ignores Caston's use of the \emph{De Insomniis} passage, cited by Ganson, to argue that Aristotle is not a proto-disjunctivist. So I take it that there is important unfinished business here. Rather, I ignore the first problem since taking a full account of it would require reading the passage in the context of \emph{De Insomniis}, staking out a position on how best to interpret \emph{phantasia}, and how \emph{phantasia} potentially explains perceptual appearances (such as the sun looking a foot long)---vexed interpretive questions all. 

In his concluding remarks Ganson sums up the second problem by claiming that Aristotle assimilates perception to thought. Beginning there will allow me to frame how I am thinking of these issues. This is important since my remarks about the role of perception in explaining unforced animal movement are relatively minor but gain in significance when placed into context.

I am among the readers of \emph{De Anima} who read it as being in dialogue with, among other things, the \emph{Theaetetus}. One of the important things being negotiated is the perception--cognition distinction. We moderns are relatively comfortable with that distinction---even if a substantive characterization of it, about which there is widespread non-collusive agreement, remains elusive. Indeed, even if some among us are skeptical about the very distinction---at least we have some idea of what they are skeptical about. Matters were, of course, different in the ancient world, where there was considerable unclarity about the distinction, if it was even drawn. Thus Aristotle complains of some of his predecessors that they assimilated thought to perception. Plato draws a distinction between perception and cognition in the \emph{Theaetetus}. It is natural that he does, given that part of the task of that dialogue is to argue against an imaginatively elaborated Protagorean assimilation of knowledge to perception. Elements of Plato's discussion can be found throughout \emph{De Anima}. However, while Aristotle accepts with Plato that a distinction should be marked between perception and cognition, that distinction is nevertheless transformed in his hands.

In the \emph{Theaetetus} 184 e 8--185 a 3, a sensory capacity is the capacity to, in Peripatetic vocabulary, present the proper sensibles of the given modality. So sight is the capacity to present color, audition the capacity to present sound, olfaction the capacity to present odor, and so on. Plato links objects being perceptible to one sense alone to a conception of the senses as powers or capacities. There are a couple of separable ideas here that get elaborated and differently developed in \emph{De Anima}: that powers or capacities are individuated by their proper exercise and that the proper exercise of sensory capacities is the presentation of their proper objects in sensory awareness. These two claims in conjunction with the claims about the proper objects of vision, audition, and olfaction imply that sight just is the capacity to see color, audition the capacity to hear sound, and olfaction the capacity to smell odor. 

Aristotle broadens the range of things which are perceptible. Proper sensibles may be perceptible in themselves---in possessing a color, say, an object contains within in itself the power of its visibility, but they are not the only things that are perceptible in themselves. So too are the common sensibles, objects not only perceptible in themselves but to more than one sensory modality. Significantly, Aristotle maintains that the difference between color and sound is perceptible. Whereas, Plato insisted that difference is an intelligible feature discovered by reason, Aristotle insists that the difference between the proper sensibles at least is perceptible.

Confining our attention to the proper sensibles, Aristotle holds with Plato that they are perceptible in themselves and perceptible to one sensory modality alone. This is why they can individuate the sensory modalities. But Aristotle in his definition of the proper sensibles attributes to them a further feature---that no error is possible about the presence of the proper sensibles. One striking thing about this feature is its negative characterization, since there are two ways to understand it. No error may be possible either in the sense that:
\begin{enumerate}
	\item perceptions of proper sensibles are always true or correct;
	\item perceptions of proper sensibles are not the kind of thing that can be true or false, correct or incorrect.
\end{enumerate}
If the perception of proper sensibles were always true or correct, then no error would be possible, at least about their presence. If, however, the perception of proper sensibles were not the kind of thing that so much as could be true or false, correct or incorrect, no error would be possible, but in a different sense. The sensing of proper sensibles would be impervious to error not because of some guarantee that the proper sensibles of a given sense falls within its ken but because the sensing of proper sensibles is fails to be evaluable as correct or incorrect.

While Aristotle's usual formulation in Book \textsc{ii} of \emph{De Anima} is that no error is possible about the presence of primary objects\index{perception!object of!primary}, he does sometimes say, especially in Book \textsc{iii} that the perception of primary objects is always true. This provides \emph{prima facie} support for the first interpretation. On this interpretation, sense perception has something like an intentional or representational content. It is at least evaluable as true or false, correct or incorrect. This is what Ganson described as the content view.  Against this suggestion, an advocate of the second interpretation might claim that, by itself, this leaves unexplained what needs explaining---\-Aristotle's apparent preference for the negative characterization in Book \textsc{ii}. Aristotle's preference for the negative characterization is well explained by the second interpretation. On that interpretation, the denial of the possibility of error is not consistent with perceptions being always true, and so the condition could only be expressed by the negative characterization. So while the first interpretation must explain Aristotle's preference for the negative characterization in Book \textsc{ii} as well as provide some guarantee for why the proper sensibles are always correctly represented in sensory experience, the second interpretation faces the potential embarrassment of explaining away the claim that the perception of proper sensibles is always true as merely loose talk if not indeed a slip on Aristotle's part.

We can decide between these rival interpretations by considering Aristotle's account of error (\emph{De Anima} \textsc{iii} 3 428\( ^{b} \)17-26, 430\( ^{a} \)27--430\( ^{b} \)5). Here too it is plausible that Aristotle has the \emph{Theaetetus} in mind, especially the puzzles about the possibility of error that animate the wax and bird cage analogies. According to Aristotle, error requires a certain kind of complexity, a complexity that the sensory presentation of the proper sensibles lacks. Specifically, only with combination is error possible:
\begin{quote}
	\ldots\ where the alternative of true or false applies, there we always find a sort of combining of objects of thought in a quasi-unity. As Empedocles said that ``where heads of many a creature sprouted without necks'' they afterwards by Love’s power were combined, so here too objects of thought which were separate are combined \ldots\ (Aristotle, \emph{De Anima} \textsc{iii} 6 430\( ^{a} \)27--32; Smith in \citealt[54]{Barnes:1984uq})
	
	For falsehood always involves a combining; for even if you assert that what is white is not white you have combined not-white. (Aristotle, \emph{De Anima} \textsc{iii} 6 430\( ^{b} \)1--3; Smith in \citealt[54]{Barnes:1984uq})
\end{quote}
(See also \emph{Categoriae} \textsc{ii} 1\( ^{a} \)16, \emph{De Interpretatione} \textsc{i} 16\( ^{a} \)9--18, \textsc{v} 17\( ^{a} \)17--20, and the \emph{Sophist} 262.) The simple presentation of the white of the sun, when not combined with other sensible elements of the scene, is not in error. But not because of any guarantee that color perception is always true. Rather, it is only when sensible objects are combined that the senses may mislead. We cannot be mistaken about the presence of the sun's whiteness upon seeing it, but we can be mistaken about the location of the whiteness, when we combine whiteness, a primary object, with other sensibles, such as location, in this case, a common sensible. Since the sensory presentation of proper sensibles does not involve combination, and combination is necessary for error, then no error is possible about the presence of these sensory objects in the strong sense that their perception is not the kind of thing that so much as could be evaluable as true or false, correct or incorrect. We simply confront what is presented to us in sensory consciousness. 

This is the basis for the second contrast that Aristotle draws between perception and understanding in the following passage from Book \textsc{iii} of \emph{De Anima}:
\begin{quote}
	That perceiving and understanding are not identical is therefore obvious; for the former is universal in the animal world, the latter is found in only a small division of it. Further, thinking is also distinct from perceiving---I mean that in which we find rightness and wrongness---right\-ness in understanding, knowledge, true opinion, wrongness in their opposites; for perception of the special objects of sense is always free from error, and is found in all animals, while it is possible to think falsely as well as truly, and thought is found only where there is discourse of reason. (Aristotle, \emph{De Anima} \textsc{iii} 3 427\( ^{b} \)7--15; Smith in \citealt[49]{Barnes:1984uq})
\end{quote}
All animals perceive, but not all animals are rational. Rational activity, such as thinking, is evaluable as correct or incorrect. But perceptions of proper sensibles, being simple presentations of these sensory objects, are insusceptible to error in this way. The line of reasoning behind this way of contrasting perception and understanding can be found in the \emph{Theaetetus}, on at least some interpretations. So it is possible that the second condition on being a primary object itself derives from Aristotle's reading of the \emph{Theaetetus} as well.

The second interpretation, according to which error is not possible about the presentation of the proper sensibles since sensory presentation is not the kind of thing that is true or false, correct or incorrect, still faces the potential embarrassment that Aristotle in Book \textsc{iii} describes perception as being always true. Allow me to make two brief remarks. First, attention to context help. In the context in which perception, of the proper sensibles at least, is described as always true the emphasis is on the contrast with judgment, which is not always true thus leading to Aristotle's quietly taking up the puzzle from the \emph{Theaetetus} in providing his own account of the possibility of error. The whole thrust of the passage is to impute to judgment the possibility of error in a way that contrasts with perception. Second, perhaps describing perception, of at least the proper sensibles, as always true, without subscribing to the content view is awkward and potentially misleading. But if we bear in mind the intellectual context, where the very distinction between perception and cognition was being forged, it is not surprising that descriptions of perception aren't as regimented as they would be in an intellectual context where the distinction between perception and cognition has become, as it were, part of normal science.

Consider, now,  Ganson's discussion of primitive agency. According to Ganson, perception allows an animal to have in mind, as the target of primitive agency, something which is not, in fact, present. Moreover, this kind of puzzle about presence in absence can motive an ascription of intentional or representational content. Honeybees, according to Aristotle, lack reason and imagination. That means that the cognitive component of their agency is limited to perception. Bees pursue distant flowers, the source of food, by their fragrant smell. So if we can find a kind of presence in absence in the olfactory perception of bees, that is some reason for thinking that perception has an intentional or representational content. 

Aristotle uses this kind of puzzle or \emph{aporia} about presence in absence to argue for the intentional character of memory (\emph{De Memoria et Reminiscentia} 450a25–451a1, for discussion see \citealt{Sorabji:2004qa}). Aristotle's response to the puzzle is to straightforwardly accept the claim of absence and reinterpret what purported to be a presentation instead as a kind of re-presentation. When one remembers Corsicus in his absence one contemplates a \emph{phantasma} caused by a previous perception of Corsicus and one conceives of the \emph{phantasma} as a likeness and reminder of Corsicus as he was perceived. Ganson's own response to the present puzzle about presence in absence follows this Peripatetic model.

According to Ganson, the puzzle arises in the following manner. How does the bee move towards the distant flower by smell? Just as we hear sounds and their sources, Ganson maintains that bees smell, not only odors, but the odorous. The bee smells, not only the odor, but the distant source of the odor, the fragrant flower. But the pleasant odor, by itself, won't prompt movement. It is already present. What prompts movement is the flower smelled. Insofar as Ganson maintains that the target of primitive agency must be cognized by perception, imagination or reason, then since the target of the bees' agency is the flower, and since bees lack imagination and reason, the flower itself must be the object of perception. But the odorous flower is not present in the way that its odor is, and yet it is smelled. This kind of presence in absence is meant to provide at least some reason to think that the odorous is part of the intentional or representational content of the bee's perceptual experience. I have four worries about Ganson's argument in ascending order of seriousness.

First, I can find no textual evidence that directly supports Gansons' contention that bees perceive not only odors but the odorous. By the same token I can find no textual evidence that directly contradicts it either. 

Second, given the role that the perception of the odorous plays in primitive agency, a worry may arise, however. The odorous is more important practically than any odors they may produce, at least potentially. Perceptually discriminating a distal flower, a source of food, in the bee's olfactory experience, is more important, from a practical point of view, than enjoying its pleasant odor. Perceiving a source of food provides a selective advantage in the way that being subject to a pleasant sensation need not. If that's right, then there is some pressure to think that olfaction is for the sake of smelling the odorous. But the odorous is a common sensible and perceptual capacities are, for Aristotle, for the sake of perceiving their proper objects (Metaphysics \( \Theta \) 8 1050\( ^{a} \)10). Postulating the odorous along with odors as the objects of olfaction threatens the explanatory framework of \emph{De Anima} of explaining perceptual capacities in terms of perceptual activities and perceptual activities in terms of their proper objects, whose perception is that for the sake of which the animal has that capacity. Though, perhaps this is Aristotle's problem and not Ganson's.

Third, the perception of the odorous is not necessary to explain how an animal, lacking imagination and reason, can follow a scent trail. All that is needed is a perceived difference in intensity of the odor, or perhaps its pleasantness, as the animal approaches the odorous. The pleasurable fragrance being more intense in a certain direction can prompt an animal to move in that direction without any representation whatsoever of the odorous. This third worry combines with the first. If there is no direct textual evidence for bee's perception of the odorous, and it is not indispensable to the explanation of an animal's ability to follow a scent trail, then why make the attribution? It can seem groundless. And given the second problem, it can seem, not only groundless, but of doubtful coherence the way it threatens the explanatory framework of \emph{De Anima}.

Fourth, there is a very specific sense in which the odorous is absent when the bee smells its odor at a distance from it. The odorous, in the present instance, the flower, is absent in the specific sense of not being spatially present. But does that really conflict with the sense in which the odorous would be present in perceptual experience, assuming that the odorous is in fact perceived? Ganson's claim that it does strikes me as turning on the same conflation that Arnauld attributes to Malebranche. If that's right, then there is no genuine conflict and no puzzle to motivate the attribution of intentional content. In \emph{Recherche de la Vérité} iii ii 1 1, Malebranche argues that if the sun and the stars were the immediate objects of perception, the soul would have to leave the body to wander about the heavens. Only in this way could the distant heavenly bodies be intimately joined with the soul. In \emph{Des Vrayes et des Fausses Idées}, Arnauld charges that Malebranche’s reasoning, here, turns on a conflation. To be sure, when we perceive an object, it is present in our perceptual experience. But the thought that the soul would have to wander the heavens to perceive the sun and the stars follows only on a different and specifically spatial understanding of presentation. But perceptual presentation is not spatial presentation. Arnauld emphasizes this denial by echoing Aristotle (\emph{De Anima} 2 7 419\( ^{a} \)13--14): ``The object must be absent from the eye, since it must be some distance from it, for what is in the eye or too close to it cannot be seen'' (Arnauld, \emph{Des Vrayes et des Fausses Idées}, Chapter 4; \citealt[62]{Gaukroger:1990rz}). Arnauld's response to Malebranche is compelling and echoes what I described in my book as Aristotle’s rejection of the Empedoclean principle \citep[chapter 2]{Kalderon:2015fr}. 


% One prima facie difficulty that Ganson raises for my interpretation concerns the phenomenal character of an animal's experience of the pleasant or unpleasant. The phenomenal character of the pleasant or the unpleasant could not be inherited from the perceiver-independent character of the object of perception since the pleasant and the unpleasant are perceiver-dependent. Even granting the prima facie character of the challenge, I wonder if Ganson is perhaps moving too quickly here. On the interpretation I offered, the proper sensibles are not only the ultimate efficient cause of their perception, not only are they their final cause---sight is for the sake of seeing color in light and the luminous in dark, but the proper sensible are also the formal cause of their perception---the conscious character of the perceptual experience depends upon and derives from the sensible form presented in it. Conscious experience formally assimilates to the sensible object presented in it. Hence Aristotle's definition of perception as the capacity to become like its object. However, whereas the exercise of our perceptual capacities may involve the perceiver or perhaps their experience becoming like its object, it does not become exactly like that object. The proper object of perception may constitutively shape its perception but it only ever does so relative to the perceiver's partial perspective. A perceiver's partial perspective may be constituted in part by facts about the constitution of their sensory organs. The same scene would appear differently to an animal with dark eyes than to animal with light eyes. So the dependency of the phenomenal character of the pleasant and the unpleasant upon the constitution of the animal's sense organ is not yet an objection the perception's formal assimilation to its object so long as this perceiver-dependency is reasonably understood as an aspect of perspectival relativity.

Ganson is keen to emphasize, and rightly so, that Aristotle is careful not to overintellectualize the cause of animal movement. Nevertheless, Ganson thinks that perception needs a content similar to the content of thought in order for perception to play the required role in explaining unforced animal movement. I concede that that is an attractive thought if one is inclined to attribute the content view to Aristotle. What I am failing to see, just yet, is why we are forced to concede that perception must have a content in the required sense. On the content view, there is a kind of sensory predication analogous with the predication involved in the corresponding perceptual judgment. Moreover, the pleasant would have a kind of universal status in that they are predicated of many things (this odor as well as that odor, say) (\emph{De Interpretatione} 7 17 a 37--8). The pleasant would be said of an object but not be in the object (\emph{Categoriae} 2 1 a 20--1 b 9). But this seems inconsistent with Aristotle's insistence that the objects of perception be particular and not universal (\emph{De Anima} 2 5 417 b 18--26). Moreover, it is not clear why, apart from an attachment to the content view, any of this is needed. Bees may lack thought, but perceiving the pleasantness of an odor may prompt unforced movement on their part given bee appetites, without their experience involving anything like sensory predication. Assertion has force. Perception is like assertion in that a presentational phenomenology has a kind of force too. But perception lacks anything like assertoric force attaching to its content. Perception has a force simply in the lively and vivid manner in which it presents its object. The bees are moved, in an unforced manner, given the nature and content of their appetites, simply by the presentation of the pleasant odor.











\bibliographystyle{plainnat}
\bibliography{Philosophy}

\end{document}
